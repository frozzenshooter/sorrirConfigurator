\chapter{Conclusion}

In summary, the result of this project is a configurator that provides various functions. Using the wizard, you can navigate back and forth in the configuration and make adjustments accordingly. The configuration as such consists of several individual steps, such as the import of an existing configuration or the configuration of subcomponents. The import of an existing configuration is validated with a schema file to enforce a valid configuration in the application. Based on the subcomponents, degradation levels can be defined and inserted into a DAG for the degradations or upgrades using drag and drop. For each defined degradation or upgrade, one can specify the states of the involved degradation levels.

The code allows an easy customization with new additional steps, which also facilitates integration of new React components. For the future, further adjustments are possible, for example, an extension for the import validation with additional properties, for example, to prevent cycles in the degradation level DAG. Another adjustment could be the exchange of the syntax highlighter, because it renders many HTML tags and it might slow down for large configuration files.
