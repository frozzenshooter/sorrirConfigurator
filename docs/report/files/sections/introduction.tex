\chapter{Introduction}
\label{sec:intro}

More and more IoT (Internet of Things) devices are finding their way into our daily lives, making it possible to combine a wide variety of end devices and sensors. Such combinations allow the design of new composite systems that combine the functionalities of the subcomponents to solve new challenges. A combined system can encounter internal problems, for example, when a subcomponent no longer operates properly. The system has to react to this and, for example, cease functionality or operate in another way.

This is where the SORRIR project comes in, in order to develop a self-organising, resilient execution platform for IoT services. To make such a system resilient, i.e. resistant to errors of the subcomponents of the system, different levels were introduced in which such a system can operate. 

A system that is composed of several subcomponents may be forced to change the level if, for example, one of these subcomponents is no longer functional. In the worst case, up to the complete deactivation of the system. In order to prevent a system from going directly into the deactivated state whenever an error occurs, various intermediate levels, the so-called degradation levels, can be defined. For example, a degradation level can continue to provide some of the original functionality, but with some limitations depending on the not-working subcomponent.

An example for such a system is a smart barrier in a parking garage, which in the ideal case recognizes a car by means of a camera and is opened automatically. The ideal case means that all subcomponents are working correctly. If, for some reason, the camera failed, the system would no longer be able to function. However, with the introduction of degradation levels, a part of the functionality could be guaranteed. A fallback for the system could be that the barrier can be manually opened with an ID card. To configure such degradation levels a JSON file is required which defines the different degradation levels and the transitions between them. For example, what should happen when a certain subcomponent is no longer functional.

This is the starting point for this project, which is about creating a graphical interface in the form of a web application. This web application will be used to configure a system with the different degradation levels and finally export it as a JSON file. In order to change or create a configuration, several components for the web application are required, like the definition of subcomponents or the import of an existing configuration. 

This report starts with a short description of the technology stack, chapter \ref{sec:tech_stack}, that was used to implement the web application. It is followed by the architecture, chapter \ref{sec:architecture}, that includes a description of the general components, their usage, and some additional information about the implementation. The last section is the conclusion that contains a short summary as well as potential further enhancements.