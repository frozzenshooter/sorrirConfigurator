\section{Architecture}
\label{sec:architecture}
\open{Use figures for the different sections (e.g. import successful and unsuccessful)}

\subsection{General concepts}
\begin{itemize}
    \item Modular
    \item Extensible ( integration of other steps)
\end{itemize}

\subsection{Wizard}
\begin{itemize}
    \item Component to allow a step by step configuration
    \item Different configurations - with and without import in the app start possible
    \item Restart of the wizard in every step possible (will delete all changes and return to the start)
    \item Easy extensible: add new steps - internally called views by:
    \begin{itemize}
        \item Extending the View- enum with the new view ("src\textbackslash util\textbackslash Views.ts")
        \item Providing a Label (will be used as caption) ("src\textbackslash util\textbackslash ViewLabelResolver.ts")
        \item Adding the View (react component) that will be displayed for the step in the Wizard ("src\textbackslash components\textbackslash wizard\textbackslash ViewSelector\textbackslash ViewSelector.tsx")
    \end{itemize}
\end{itemize}

\subsection{Import}
\begin{itemize}
    \item Step to import an existing configuration (json file)
    \item Will validate the Json file and display errors (validates if it can be parsed and will also display if it matches the formal description of the configuration ( details in the validation section) 
    \item After a successful import - internal configuration state will be updated and the current state will be displayed in a json viewer (with syntax highlight)
    \item internal configuration is based on a several interfaces found in "src\textbackslash models"
\end{itemize}

\subsection{Validation}
\begin{itemize}
    \item Validates if the file can be parsed as Json element
    \item If it can be parsed the validation will be done
    \item short explanation of the different schema types?
    \item use of ajv ( support of JSON Schema and JSON Type Definition)
    \item atm most tools support Json Schema (cite \url{https://ajv.js.org/guide/schema-language.html})  $\rightarrow$ selection for our usecase - but no offical RFC only internet draft
    \item ajv used because it supports both - migration later possible (to RFC8927 JSON Type Definition)
    \item Short overview of the implemented schema for the configuration (perhaps using UML diagram)
\end{itemize}

\subsection{Subcomponent configuration}
\begin{itemize}
    \item Subcomponents required for the creation of degradation levels
    \item Subcomponent has an id, a name and several shadowmodes (internal state of the subcomponent)
    \item each shadowmode has an id and a name 
    \item In order to be able to manage the relevant subcomponents for the component, a table with a creation/edit and deletion dialog is offered
    \item The creation/edit dialog offers several functions: 
        \begin{itemize}
            \item Internal validation function (non empty fields + check for already existing subcomponents with the same id)
            \item id autofill ( based on the name) if no value is set
        \end{itemize}
\end{itemize}

\subsection{Degradation Level Management}
\begin{itemize}
    \item Creation/Deletion in the Degradation and Upgrade Configuration views
    \item Create/Edit with a Dialog:
    \begin{itemize}
        \item name and id as string
        \item the subcomponent states of the level (will show all possible values of a subcomponent and to choose what has to be true for this level)
        \item creation of internal states of the Level (with custom chip input)
        \item validation if id exists/ etc.
    \end{itemize}
    \item Multi selection in the Degradation and Upgrade Configuration View
\end{itemize}

\subsection{Degradation and Upgrade configuration}
\begin{itemize}
    \item Custom Graph View for the Hierarchy (Recursive generation of the Graph)
    \item Drag and Drop Support to update the Graph (degradation level hierarchy)
    \item Off state as default node
    \item Explain model on how to save the hierarchy (LevelChange model) 
    \item shows all levels even the ones that are not inserted in the hierarchy yet
    \item Upgrade: Inverse of the Degradation Graph - same functionality as Degradation Graph
    \item Both will be saved separately in the configuration and are also separate steps in the wizard
\end{itemize}

\subsection{LevelChange configuration}
\begin{itemize}
    \item configure the state in which the level after the degradation or upgrade is
    \item separate step in the wizard for both degradation and uprade
    \item shows for all LevelChanges the states of the corresponding level and allows the selection of the state the level will be in after the degradation/upgrade 
\end{itemize}